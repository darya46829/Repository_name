\documentclass[a4paper,12pt]{article}

\usepackage[english, russian]{babel}
\usepackage[T2A]{fontenc}
\usepackage[utf8]{inputenc}

\usepackage{amsthm, amsfonts, amssymb, amsthm, mathtools, dsfont}

\theoremstyle{plain}
\newtheorem{theorem}{Теорема}
\newtheorem*{lemma}{Лемма}
\newtheorem*{corollary}{Следствие}
\newtheorem*{definition}{Определение}
\newtheorem*{statement}{Утверждение}
\newtheorem*{examples}{Примеры}


\begin{document}
	\title{Группы и алгебры Ли}
	\date{Весенний семестр 2024 г.}
	\maketitle{}
	\vspace{14cm}
	\section{Предварительные сведения}
	Под $K$ мы понимаем $\mathds{R}$ или $\mathds{C}$.
	
	\begin{definition} Гладкое многообразие размерности $n$ над $K$"--- \end{definition}
	
	\begin{definition} Прямое произведение гладких многообразий $X$ и $Y$"--- \end{definition}
	
	\begin{definition} Гладкое отображение между гладкими многообразиями $X$ и $Y$ "--- \end{definition}
	
	\begin{statement} Корректность определения(гладкость отображения не зависит от выбора атласов). \end{statement}
	Доказательство. $\blacklozenge$
	
	\begin{statement} Композиция гладких отображений гладкая. \end{statement}
	Доказательство. $\blacklozenge$
	
	\begin{definition} $k$-подмногообразие $Y$ многообразия $X$ "--- \end{definition}
	
	\begin{definition} Касательное пространство $T_x X$ к $n$-мерному многообразию $X$ в точке $x$ "--- \end{definition}
	
	\begin{theorem} Корректность определения(независимость от карты). Касательное пространство в точке для фиксированной карты изоморфно $\mathds{R}^n$. Пространства, определяемые с помощью разных карт, получаются друг из друга невырожденным линейным опреатором $\mathds{R}^n$. \end{theorem}
	Доказательство. $\blacklozenge$
	
	\begin{definition} k-подмногообразие $Y$ гладкого многообразия $X$ "--- \end{definition}
	
	\begin{definition} Дифференциал гладкого отображения между гладкими многообразиями $X$ и $Y$ в точке $x \in X$ "--- \end{definition}
	
	\begin{theorem} Корректность определения и линейность. \end{theorem}
	Доказательство. $\blacklozenge$
	
	\begin{statement} Дифференциал композициии есть композиция дифференциалов. \end{statement}
	Доказательство. $\blacklozenge$
	
	\begin{definition} Ранг гладкого отображения между гладкими многообразиями в точке "--- ранг его дифференциала в точке. \end{definition}
	
	\subsection{подраздел}
	
	\dots
	
	\subsection*{подраздел еще}
	
	\dots
	
	\section{Группы и погруппы Ли}
	
	\begin{definition} Группа Ли над $K$ размерности $n$ "--- гладкое многообразие $G$ над $K$ размерности $n$ со структурой группы, такое что являются гладкими отображения: 
		
		1. $\mu:G\times G\rightarrow G$, $\mu$ "--- групповая операция
		
		2. $r:G\rightarrow G$, $r$ "--- операция взятия обратного
	\end{definition}
	
	\begin{examples} Примеры групп Ли:
	
	1.
		
	2.
		
	3.
		
	4.
	\end{examples}

	\begin{definition} Подгруппа Ли $H\subset G$ "--- подгруппа $G$, являющаяся подмногообразием $G$. \end{definition}
	
	\begin{lemma} $G$ "--- группа Ли, $H\leq G$, тогда $H$ "--- $k$-мерная подгруппа Ли тогда и только тогда, когда $H$ "--- $k$-мерное подмногообразие локально в точке $e$. \end{lemma}
	
\end{document}
